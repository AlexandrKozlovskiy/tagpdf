% \iffalse meta-comment
%
%% File: tagpdf-mc.dtx
%
% Copyright (C) 2019-2021 Ulrike Fischer
%
% It may be distributed and/or modified under the conditions of the
% LaTeX Project Public License (LPPL), either version 1.3c of this
% license or (at your option) any later version.  The latest version
% of this license is in the file
%
%    https://www.latex-project.org/lppl.txt
%
% This file is part of the "tagpdf bundle" (The Work in LPPL)
% and all files in that bundle must be distributed together.
%
% -----------------------------------------------------------------------
%
% The development version of the bundle can be found at
%
%    https://github.com/u-fischer/tagpdf
%
% for those people who are interested.
%<*driver>
\RequirePackage{pdfmanagement-testphase}
\DeclareDocumentMetadata{}
\makeatletter
\declare@file@substitution{doc.sty}{doc-v3beta.sty}
\makeatother
\documentclass{l3doc}
\usepackage{array,booktabs,caption}
\hypersetup{pdfauthor=Ulrike Fischer,
 pdftitle=tagpdf-mc module (tagpdf)}
\begin{document}
  \DocInput{\jobname.dtx}
\end{document}
%</driver>
% \fi
% \title{^^A
%   The \pkg{tagpdf-mc} module\\ Code related to Marked Content (mc-chunks)  ^^A
%   \\ part of the tagpdf package
% }
%
% \author{^^A
%  Ulrike Fischer\thanks
%    {^^A
%      E-mail:
%        \href{mailto:fischer@troubleshooting-tex.de}
%          {fischer@troubleshooting-tex.de}^^A
%    }^^A
% }
%
% \date{Version 0.82, released 2021-06-14}
% \maketitle
% \begin{documentation}
% \section{Public Commands}
%  \begin{function}{\tag_mc_begin:n,\tag_mc_end:}
%   \begin{syntax}
%     \cs{tag_mc_begin:n}\Arg{key-values}\\
%     \cs{tag_mc_end:}
%   \end{syntax}
% These commands insert the end code of the marked content.
% They don't end a group and in generic mode it doesn't matter
% if they are in another group as the starting commands.
% In generic mode both commands check if they are correctly nested
% and issue a warning if not.
% \end{function}
%
%  \begin{function}{\tag_mc_use:n}
%   \begin{syntax}
%     \cs{tag_mc_use:n}\Arg{label}
%   \end{syntax}
% These command allow to record a marked content that was stashed away before
% into the current structure. A marked content can be used only once --
% the command will issue a warning if an mc is use a second time.
% \end{function}
%
%  \begin{function}[added = 2019-11-20]
%   {
%     \tag_mc_artifact_group_begin:n, \tag_mc_artifact_group_end:
%   }
%   \begin{syntax}
%     \cs{tag_mc_artifact_group_begin:n} \Arg{name}\\
%     \cs{tag_mc_artifact_group_end:}
%   \end{syntax}
%   This command pair creates a group with an artifact marker at the begin
%   and the end. Inside the group the tagging commands are disabled.
%   It allows to mark a complete region as artifact without having to worry
%   about user commands with tagging commands.
%   \meta{name} should be a value allowed also for the |artifact| key.
%   It pushes and pops mc-chunks at the begin and end.
%   TODO: document is in tagpdf.tex
%   \end{function}
%
%  \begin{function}[added = 2021-04-22]
%   {
%     \tag_mc_end_push:, \tag_mc_begin_pop:n
%   }
%   \begin{syntax}
%     \cs{tag_mc_end_push:} \\
%     \cs{tag_mc_begin_pop:n}\Arg{key-values}
%   \end{syntax}
% If there is an open mc chunk,
% \cs{tag_mc_end_push:} ends it and pushes its tag of the (global) stack.
% If there is no open chunk, it puts $-1$ on the stack (for debugging)
% \cs{tag_mc_begin_pop:n} removes a value from the stack. If it is different from
% $-1$ it opens a tag with it.
% The reopened mc chunk looses info like the alttext for now.
% \end{function}
%
% \begin{function}[pTF, EXP]{\tag_mc_if_in:}
%   \begin{syntax}
%     \cs{tag_mc_if_in:TF} \Arg{true code} \Arg{false code}
%   \end{syntax}
%   Determines if a mc-chunk is open.
% \end{function}

%
% \section{Public keys}
% The following keys can be used with |\tag_mc_begin:n|, |\tagmcbegin|,
% |\tag_mc_begin_pop:n|,
%
% \begin{function}{tag}
% This key is required, unless artifact is used.
% The value is a tag like |P| or |H1| without a slash at the begin,
% this is added by the code.
% It is possible to setup new tags.
% The value of the key is expanded, so it can be a command.
% The expansion is passed unchanged to the PDF,
% so it should with a starting slash give a valid PDF name
% (some ascii with numbers like \texttt{H4} is fine).
% \end{function}
%
% \begin{function}{artifact}
%  This will setup the marked content as an artifact. The key should be used
%  for content that should be ignored.
%  The key can take one of the values |pagination|,
%  |layout|,  |page|,  |background| and |notype|
%  (this is the default).
% \end{function}
%
% \begin{function}{raw}
% This key allows  to add more entries to the properties dictionary.
% The value must be correct, low-level PDF. E.g.
% \verb+raw=/Alt (Hello)+ will insert an alternative Text.
% \end{function}
%
% \begin{function}{alttext,alttext-o}
%  This key inserts an \texttt{/Alt} value in the property dictionary of the BDC operator.
%  The value is handled as verbatim string, commands are not expanded. With |alttext-o|
%  the value is expanded once.
% \end{function}
%
% \begin{function}{actualtext,actualtext-o}
%  This key inserts an \texttt{/ActualText} value in the property dictionary
%  of the BDC operator. The value is handled as verbatim string,
%  commands are not expanded.With |actualtext-o|
%  the value is expanded once.
% \end{function}
%
% \begin{function}{label}
%  This key sets a label by which one can call the marked content
%  later in another structure
%  (if it has been stashed with the |stash| key).
%  Internally the label name will start with \texttt{tagpdf-}.
% \end{function}
%
% \begin{function}{stash}
% This \enquote{stashes} an mc-chunk: it is not inserted into the current structure.
% It should be normally be used along with a label to be able to use the mc-chunk
% in another place.
% \end{function}
% \end{documentation}
% \begin{implementation}
% The code is splitted into three parts: code shared by all engines,
% code specific to luamode and code not used by luamode.
% \section{Marked content code -- shared}
%    \begin{macrocode}
%<@@=tag>
%<*shared>
\ProvidesExplPackage {tagpdf-mc-code-shared} {2021-06-14} {0.82}
  {part of tagpdf - code related to marking chunks -
   code shared by generic and luamode }
%    \end{macrocode}
%
% \subsection{Variables and counters}
% MC chunks must be counted.
% I use a latex counter for the absolute count, so that it is added to
% |\cl@@ckpt| and restored e.g. in tabulars and align.
% |\int_new:N  \c@g_@@_MCID_int| and
% |\tl_put_right:Nn\cl@@ckpt{\@elt{g_uf_test_int}}|
% would work too, but as the name is not expl3 then too, why bother?
% The absolute counter can be used to label and to check if the page
% counter needs a reset.
%
% \begin{variable}{g_@@_MCID_abs_int}
%    \begin{macrocode}
\newcounter { g_@@_MCID_abs_int }
%    \end{macrocode}
% \end{variable}
%
% \begin{macro}{\_@@_get_mc_abs_cnt:}
% A (expandable) function to get the current value of the cnt.
%    \begin{macrocode}
\cs_new:Npn \_@@_get_mc_abs_cnt: { \int_use:N \c@g_@@_MCID_abs_int }
%    \end{macrocode}
% \end{macro}
%
% \begin{variable}{\g_@@_MCID_tmp_bypage_int}
% The following hold the temporary by page number assigned to a mc.
% It must be defined in the shared code to avoid problems with labels.
%    \begin{macrocode}
\int_new:N \g_@@_MCID_tmp_bypage_int
%    \end{macrocode}
% \end{variable}
%
% \begin{variable}{\g_@@_mc_parenttree_prop}
% For every chunk we need to know the structure it is in, to
% record this in the parent tree. We store this in a property.\\
% key:   absolute number of the mc (tagmcabs)\\
% value: the structure number the mc is in
%    \begin{macrocode}
\@@_prop_new:N \g_@@_mc_parenttree_prop
%    \end{macrocode}
% \end{variable}
%
%\begin{variable}{\g_@@_mc_parenttree_prop}
% Some commands (e.g. links) want to close a previous mc and reopen it after
% they did their work. For this we create a stack:
%    \begin{macrocode}
\seq_new:N \g_@@_mc_stack_seq
%    \end{macrocode}
% \end{variable}
%
%\begin{variable}{\l_@@_mc_artifact_type_tl}
% Artifacts can have various types like Pagination or Layout. This stored
% in this variable.
%    \begin{macrocode}
\tl_new:N \l_@@_mc_artifact_type_tl
%    \end{macrocode}
% \end{variable}
%
%\begin{variable}{\l_@@_mc_key_stash_bool,\l_@@_mc_artifact_bool}
%This booleans store the stash and artifact status of the mc-chunk.
%    \begin{macrocode}
\bool_new:N \l_@@_mc_key_stash_bool
\bool_new:N \l_@@_mc_artifact_bool
%    \end{macrocode}
% \end{variable}
%
% \begin{variable}
%  {
%   \l_@@_mc_key_tag_tl,
%   \g_@@_mc_key_tag_tl,
%   \l_@@_mc_key_label_tl,
%   \l_@@_mc_key_properties_tl
%  }
% Variables used by the keys. |\l_@@_mc_key_properties_tl|
% will collect a number of values. TODO: should this be a pdfdict now?
%    \begin{macrocode}
\tl_new:N \l_@@_mc_key_tag_tl
\tl_new:N \g_@@_mc_key_tag_tl
\tl_new:N \l_@@_mc_key_label_tl
\tl_new:N \l_@@_mc_key_properties_tl
%    \end{macrocode}
% \end{variable}
%
% \subsection{Functions}
% \begin{macro}{\_@@_mc_handle_mc_label:n}
% The commands labels a mc-chunk. It is used if the user explicitly
% labels the mc-chunk with the |label| key. The argument is the
% value provided by the user. It stores the attributes\\
% |tagabspage|: the absolute page, |\g_shipout_readonly_int|,\\
% |tagmcabs|: the absolute mc-counter |\c@g_@@_MCID_abs_int|,\\
% |tagmcid|: the ID of the chunk on the page |\g_@@_MCID_tmp_bypage_int|, this
%  typically settles down after a second compilation.
%  The reference command is defined in tagpdf.dtx and is based on l3ref.
%    \begin{macrocode}
\cs_new:Nn \_@@_mc_handle_mc_label:n
  {
    \@@_ref_label:en{tagpdf-#1}{mc}
  }
%    \end{macrocode}
% \end{macro}
%
% \begin{macro}
%   {
%     \tag_mc_artifact_group_begin:n,
%     \tag_mc_artifact_group_end:
%   }
% This opens an artifact of the type given in the argument,
% and then stops all tagging. It creates a group.
% It pushes and pops mc-chunks at the begin and end.
%    \begin{macrocode}
\cs_new_protected:Npn \tag_mc_artifact_group_begin:n #1
 {
  \tag_mc_end_push:
  \tag_mc_begin:n {artifact=#1}
  \tag_stop_group_begin:
 }

\cs_new_protected:Npn \tag_mc_artifact_group_end:
 {
  \tag_stop_group_end:
  \tag_mc_end:
  \tag_mc_begin_pop:n{}
 }
%    \end{macrocode}
% \end{macro}
%
% \begin{macro}{\tag_mc_end_push:, \tag_mc_begin_pop:n}
%
%    \begin{macrocode}
\cs_new_protected:Npn \tag_mc_end_push:
  {
    \@@_check_if_active_mc:T
      {
        \@@_mc_if_in:TF
          {
            \seq_gpush:Nx \g_@@_mc_stack_seq { \tag_get:n {mc_tag} }
            \@@_check_mc_pushed_popped:nn
              { pushed }
              { \tag_get:n {mc_tag} }
            \tag_mc_end:
          }
          {
            \seq_gpush:Nn \g_@@_mc_stack_seq {-1}
            \@@_check_mc_pushed_popped:nn { pushed }{-1}
          }
      }
  }

\cs_new_protected:Npn \tag_mc_begin_pop:n #1
  {
    \@@_check_if_active_mc:T
      {
        \seq_gpop:NNTF \g_@@_mc_stack_seq \l_@@_tmpa_tl
          {
            \tl_if_eq:NnTF \l_@@_tmpa_tl {-1}
              {
                \@@_check_mc_pushed_popped:nn {popped}{-1}
              }
              {
                \@@_check_mc_pushed_popped:nn {popped}{\l_@@_tmpa_tl}
                \tag_mc_begin:n {tag=\l_@@_tmpa_tl,#1}
              }
          }
          {
            \@@_check_mc_pushed_popped:nn {popped}{empty~stack,~nothing}
          }
      }
  }
%    \end{macrocode}
% \end{macro}
%
% \subsection{Keys}
% This are the keys where the code can be shared between the modes.
%
% \begin{macro}{stash,__artifact-bool,__artifact-type}
% the two internal artifact keys are use to define the public |artifact|.
%    \begin{macrocode}
\keys_define:nn { @@ / mc }
  {
    stash                    .bool_set:N    = \l_@@_mc_key_stash_bool,
    __artifact-bool          .bool_set:N    = \l_@@_mc_artifact_bool,
    __artifact-type          .choice:,
    __artifact-type / pagination .code:n    =
      {
        \tl_set:Nn \l_@@_mc_artifact_type_tl { Pagination }
      },
    __artifact-type / layout     .code:n    =
      {
        \tl_set:Nn \l_@@_mc_artifact_type_tl { Layout }
      },
    __artifact-type / page       .code:n    =
      {
        \tl_set:Nn \l_@@_mc_artifact_type_tl { Page }
      },
    __artifact-type / background .code:n    =
      {
        \tl_set:Nn \l_@@_mc_artifact_type_tl { Background }
      },
    __artifact-type / notype     .code:n    =
      {
        \tl_set:Nn \l_@@_mc_artifact_type_tl {}
      },
    __artifact-type /      .code:n    =
      {
        \tl_set:Nn \l_@@_mc_artifact_type_tl {}
      },
  }
%    \end{macrocode}
% \end{macro}
%    \begin{macrocode}
%</shared>
%    \end{macrocode}
% \section{Marked content code -- generic mode}
%    \begin{macrocode}
%<*generic>
\ProvidesExplPackage {tagpdf-mc-code-generic} {2021-06-14} {0.82}
 {part of tagpdf - code related to marking chunks - generic mode}
%    \end{macrocode}
% \subsection{Variables}
% \begin{variable}{\g_@@_in_mc_bool}
% This booleans records if a mc is open, to test nesting.
%    \begin{macrocode}
\bool_new:N \g_@@_in_mc_bool
%    \end{macrocode}
% \end{variable}

% \begin{variable}{\g_@@_MCID_byabspage_prop}
% This property will hold the current maximum on a page
% it will contain key-value of type \meta{abspagenum}=\meta{max mcid}
%
%    \begin{macrocode}
\@@_prop_new:N \g_@@_MCID_byabspage_prop
%    \end{macrocode}
% \end{variable}

% \begin{variable}{\l_@@_mc_ref_abspage_tl}
% We need a ref-label system to ensure that the MCID cnt
% restarts at 0 on a new page
% This will be used to store the tagabspage attribute retrieved from
% a label.
%    \begin{macrocode}
\tl_new:N \l_@@_mc_ref_abspage_tl
%    \end{macrocode}
% \end{variable}

% \begin{variable}{\l_@@_mc_tmpa_tl}
% temporary variable
%    \begin{macrocode}
\tl_new:N \l_@@_mc_tmpa_tl
%    \end{macrocode}
% \end{variable}

% \subsection{Functions}

% \begin{macro}[pTF]{\@@_mc_if_in:,\tag_mc_if_in:}
% This is a test if a mc is open or not. It depends simply on a global boolean:
% mc-chunks are added linearly so nesting should not be relevant.
%    \begin{macrocode}
\prg_new_conditional:Nnn \@@_mc_if_in: {p,T,F,TF}
  {
    \bool_if:NTF \g_@@_in_mc_bool
      { \prg_return_true:  }
      { \prg_return_false: }
  }

\prg_new_eq_conditional:NNn \tag_mc_if_in: \@@_mc_if_in: {p,T,F,TF}
%    \end{macrocode}
% \end{macro}
%
% \begin{macro}{ \@@_mc_bmc:n,\@@_mc_emc:,\@@_mc_bdc:nn,\@@_mc_bdc:nx}
% These are the low-level commands. There are now equal to the
% pdfmanagement commands generic mode, but we use an indirection
% in case luamode need something else.
% change 04.08.2018: the commands do not check the validity of the arguments or try
% to escape them, this should be done before using them.
%    \begin{macrocode}
% #1 tag, #2 properties
\cs_set_eq:NN \@@_mc_bmc:n  \pdf_bmc:n
\cs_set_eq:NN \@@_mc_emc:   \pdf_emc:
\cs_set_eq:NN \@@_mc_bdc:nn \pdf_bdc:nn
\cs_generate_variant:Nn \@@_mc_bdc:nn {nx}
%    \end{macrocode}
% \end{macro}
%
% \begin{macro}
%   {
%     \@@_mc_bdc_mcid:nn,\@@_mc_bdc_mcid:n,
%     \@@_mc_handle_mcid:nn,\@@_mc_handle_mcid:VV
%   }
%
% This create a BDC mark with an |/MCID| key. Most of the work here is to get
% the current number value for the MCID: they must be numbered by page
% starting with 0 and then successively.
% The first argument is the tag, e.g. |P| or |Span|, the second is used to pass
% more properties.
% We also define a wrapper around the low-level command as luamode will need
% something.
%    \begin{macrocode}
\cs_new_protected:Npn \@@_mc_bdc_mcid:nn #1 #2
  {
    \int_gincr:N \c@g_@@_MCID_abs_int
    \tl_set:Nx \l_@@_mc_ref_abspage_tl
      {
        \@@_ref_value:enn %3 args
          {
            mcid-\int_use:N \c@g_@@_MCID_abs_int
          }
          { tagabspage }
          {-1}
      }
    \prop_get:NoNTF
      \g_@@_MCID_byabspage_prop
      {
        \l_@@_mc_ref_abspage_tl
      }
      \l_@@_mc_tmpa_tl
      {
        %key already present, use value for MCID and add 1 for the next
        \int_gset:Nn \g_@@_MCID_tmp_bypage_int { \l_@@_mc_tmpa_tl }
        \@@_prop_gput:Nxx
          \g_@@_MCID_byabspage_prop
          { \l_@@_mc_ref_abspage_tl }
          { \int_eval:n {\l_@@_mc_tmpa_tl +1} }
      }
      {
        %key not present, set MCID to 0 and insert 1
        \int_gzero:N \g_@@_MCID_tmp_bypage_int
        \@@_prop_gput:Nxx
          \g_@@_MCID_byabspage_prop
          { \l_@@_mc_ref_abspage_tl }
          {1}
      }
    \@@_ref_label:en
      {
        mcid-\int_use:N \c@g_@@_MCID_abs_int
      }
      { mc }
     \@@_mc_bdc:nx
       {#1}
       { /MCID~\int_eval:n { \g_@@_MCID_tmp_bypage_int }~ \exp_not:n { #2 } }
 }
\cs_new_protected:Npn \@@_mc_bdc_mcid:n #1
  {
    \@@_mc_bdc_mcid:nn {#1} {}
  }

\cs_new_protected:Npn \@@_mc_handle_mcid:nn #1 #2 %#1 tag, #2 properties
  {
    \@@_mc_bdc_mcid:nn {#1} {#2}
  }

\cs_generate_variant:Nn \@@_mc_handle_mcid:nn {VV}
%    \end{macrocode}
% \end{macro}

% \begin{macro}{\@@_mc_handle_stash:n}
% This is the handler which puts a mc into the
% the current structure. The argument is the number of the mc.
% Beside storing the mc into the structure, it also has to record the
% structure for the parent tree.
% The name is a bit confusing, \ldots.
%    \begin{macrocode}
\cs_new_protected:Npn \@@_mc_handle_stash:n #1 %1 mcidnum
  {
    \@@_check_mc_used:n {#1}
    \@@_struct_kid_mc_gput_right:nn
      { \g_@@_struct_stack_current_tl }
      {#1}
   \prop_gput:Nxx \g_@@_mc_parenttree_prop
     {#1}
     { \g_@@_struct_stack_current_tl }
  }
%    \end{macrocode}
% \end{macro}

% \begin{macro}
%  {
%    \@@_mc_bmc_artifact:,
%    \@@_mc_bmc_artifact:n,
%    \@@_mc_handle_artifact:N
%   }
% Two commands to create artifacts, one without type, and one with.
% We define also a wrapper handler as luamode will need a different definition.
% TODO: perhaps later: more properties for artifacts
%    \begin{macrocode}
\cs_new_protected:Npn  \@@_mc_bmc_artifact:
  {
    \@@_mc_bmc:n {Artifact}
  }
\cs_new_protected:Npn \@@_mc_bmc_artifact:n #1
  {
    \@@_mc_bdc:nn {Artifact}{/Type/#1}
  }
\cs_new_protected:Npn \@@_mc_handle_artifact:N #1
   % #1 is a var containing the artifact type
  {
    \tl_if_empty:NTF #1
      { \@@_mc_bmc_artifact: }
      { \exp_args:NV\@@_mc_bmc_artifact:n #1 }
  }
%    \end{macrocode}
% \end{macro}

% \begin{macro}{ \@@_get_data_mc_tag: }
% This allows to retrieve the active mc-tag.
% It is use by the get command.
%    \begin{macrocode}
\cs_new:Nn \@@_get_data_mc_tag: { \g_@@_mc_key_tag_tl }
%    \end{macrocode}
% \end{macro}
%
% \begin{macro}{\tag_mc_begin:n,\tag_mc_end:}
% These are the core public commands to open and close an mc.
% They don't need to be in the same group or grouping level,
% but the code expect that they are issued linearly. The tag and
% the state is passed to the end command through a global var and
% a global boolean.
%    \begin{macrocode}
\cs_new_protected:Npn \tag_mc_begin:n #1 %#1 keyval
  {
    \@@_check_if_active_mc:T
      {
        \group_begin: %hm
        \@@_check_mc_if_nested:
        \bool_gset_true:N \g_@@_in_mc_bool
        \keys_set:nn { @@ / mc } {#1}
        \bool_if:NTF \l_@@_mc_artifact_bool
          { %handle artifact
            \@@_mc_handle_artifact:N \l_@@_mc_artifact_type_tl
          }
          { %handle mcid type
            \@@_check_mc_tag:N  \l_@@_mc_key_tag_tl
            \@@_mc_handle_mcid:VV
               \l_@@_mc_key_tag_tl
               \l_@@_mc_key_properties_tl
            \tl_if_empty:NF {\l_@@_mc_key_label_tl}
              {
                \exp_args:NV
                \@@_mc_handle_mc_label:n \l_@@_mc_key_label_tl
              }
            \bool_if:NF \l_@@_mc_key_stash_bool
              {
                \@@_mc_handle_stash:n { \int_use:N \c@g_@@_MCID_abs_int }
              }
          }
        \group_end:
      }
  }
\cs_new_protected:Nn \tag_mc_end:
  {
    \@@_check_if_active_mc:T
      {
        \@@_check_mc_if_open:
        \bool_gset_false:N \g_@@_in_mc_bool
        \tl_gset:Nn  \g_@@_mc_key_tag_tl { }
        \@@_mc_emc:
      }
  }
%    \end{macrocode}
% \end{macro}

% \begin{macro}{\tag_mc_use:n}
% These command allow to record a marked content that was stashed away before
% into the current structure. A marked content can be used only once --
% the command will issue a warning if an mc is use a second time.
% The argument is a label name set with the |label| key.
%
% TODO: is the claim about the warning right???
% TODO: is testing for struct the right test?
%    \begin{macrocode}
\cs_new_protected:Npn \tag_mc_use:n #1 %#1: label name
  {
    \@@_check_if_active_struct:T
      {
        \tl_set:Nx  \l_@@_tmpa_tl { \@@_ref_value:enn{tagpdf-#1}{tagmcabs}{} }
        \tl_if_empty:NTF\l_@@_tmpa_tl
          {
            \msg_warning:nnn {tag} {mc-label-unknown} {#1}
          }
          {
            \prop_gput:Nxx
              \g_@@_mc_parenttree_prop
              {
                \l_@@_tmpa_tl
                %\@@_ref_value:enn {tagpdf-#1} {tagmcabs} {}
              }
              {
                \g_@@_struct_stack_current_tl
              }
            \@@_struct_kid_mc_gput_right:nn
              {
                \g_@@_struct_stack_current_tl
              }
              {
                \l_@@_tmpa_tl
                %\@@_ref_value:enn {tagpdf-#1} {tagmcabs} {}
              }
           }
       }
  }
%    \end{macrocode}
% \end{macro}
%
% \subsection{Keys}
% Definitions are different in luamode.
% |tag| and |raw| are expanded as |\lua_now:e| in lua does it too and
% we assume that their values are safe.
% \begin{macro}
%  {
%   tag,raw,
%   alttext,alttext-o,
%   actualtext,actualtext-o,
%   label,artifact
%  }
%    \begin{macrocode}
\keys_define:nn { @@ / mc }
  {
    tag .code:n = % the name (H,P,Span) etc
      {
        \tl_set:Nx   \l_@@_mc_key_tag_tl { #1 }
        \tl_gset:Nx  \g_@@_mc_key_tag_tl { #1 }
      },
    raw  .code:n =
      {
        \tl_put_right:Nx \l_@@_mc_key_properties_tl { #1 }
      },
    alttext .code:n = % Alt property
      {
        \str_set_convert:Nnon
          \l_@@_tmpa_str
          { #1 }
          { default }
          { utf16/hex }
        \tl_put_right:Nn \l_@@_mc_key_properties_tl { /Alt~< }
        \tl_put_right:No \l_@@_mc_key_properties_tl { \l_@@_tmpa_str>~ }
      },
    alttext-o .code:n      = % Alt property
      {
        \str_set_convert:Noon
          \l_@@_tmpa_str
          { #1 }
          { default }
          { utf16/hex }
        \tl_put_right:Nn \l_@@_mc_key_properties_tl { /Alt~< }
        \tl_put_right:No \l_@@_mc_key_properties_tl { \l_@@_tmpa_str>~ }
      },
    actualtext .code:n      = % ActualText property
      {
        \str_set_convert:Nnon
          \l_@@_tmpa_str
          { #1 }
          { default }
          { utf16/hex }
        \tl_put_right:Nn \l_@@_mc_key_properties_tl { /ActualText~< }
        \tl_put_right:No \l_@@_mc_key_properties_tl { \l_@@_tmpa_str>~ }
      },
    actualtext-o .code:n      = % ActualText property
      {
        \str_set_convert:Noon
          \l_@@_tmpa_str
          { #1 }
          { default }
          { utf16/hex }
        \tl_put_right:Nn \l_@@_mc_key_properties_tl { /ActualText~< }
        \tl_put_right:No \l_@@_mc_key_properties_tl { \l_@@_tmpa_str>~ }
      },
    label .tl_set:N        = \l_@@_mc_key_label_tl,
    artifact .code:n       =
      {
        \exp_args:Nnx
          \keys_set:nn
            { @@ / mc }
            { __artifact-bool, __artifact-type=#1 }
      },
    artifact .default:n    = {notype}
  }
%</generic>
%    \end{macrocode}
% \end{macro}
%
% \section{Marked content code -- luamode code}
% luamode uses attributes to mark mc-chunks. The two attributes used are
% defined in the backend file. The backend also load the lua file, as it can contain
% functions needed elsewhere.
% The attributes for mc are currently local, but this will probably change, see
% the |global-mc| option.\\
% |\newattribute \l_@@_mc_type_attr|:  the value represent the type\\
% |\newattribute \l_@@_mc_cnt_attr|:   will hold the |\c@g_@@_MCID_abs_int value|
%
% Handling attribute needs a different system to number the page wise mcid's:
% a tagmcbegin ... tagmcend pair no longer surrounds exactly one mc chunk:
% it can be split
% at page breaks. We know the included mcid(s) only after the ship out.
% So for the |struct -> mcid| mapping we
% need to record |struct -> mc-cnt| (in |\g_@@_mc_parenttree_prop| and/or a lua table
% and at shipout |mc-cnt-> {mcid, mcid, ...}|
% and when building the trees connect both.
%
% Key definitions are overwritten for luatex to store that data in tables.
% The data for the mc are in |ltx.@@.mc[absnum]|.
% The fields of the table are:\\
% tag : the type (a string)\\
% raw : more properties (string)\\
% label: a string. \\
% artifact: the presence indicates an artifact, the value (string) is the type.\\
% kids: a array of tables\\
% |{1={kid=num2,page=pagenum1}, 2={kid=num2,page=pagenum2},...}|,\\
% this describes the chunks the mc has been split to by the traversing code\\
% parent: the number of the structure it is in. Needed to build the parent tree.
%
%    \begin{macrocode}
%<*luamode>
\ProvidesExplPackage {tagpdf-mc-code-lua} {2021-06-14} {0.82}
  {tagpdf - mc code only for the luamode }
%    \end{macrocode}
% The main function which wanders through the shipout box to inject the literals.
% if the new callback is there, it is used.
%    \begin{macrocode}
\hook_gput_code:nnn{begindocument}{tagpdf/mc}
  {
    \bool_if:NT\g__tag_active_mc_bool
      {
        \lua_now:e
          {
            if~luatexbase.callbacktypes.pre_shipout_filter~then~
              luatexbase.add_to_callback("pre_shipout_filter", function(TAGBOX)~
              ltx.@@.func.mark_shipout(TAGBOX)~return~true~
              end, "tagpdf")~
            end
          }
       \lua_now:e
         {
           if~luatexbase.callbacktypes.pre_shipout_filter~then~
           token.get_next()~
           end
         }\@secondoftwo\@gobble
           {
             \hook_gput_code:nnn{shipout/before}{tagpdf/lua}
               {
                 \lua_now:e
                   { ltx.@@.func.mark_shipout (tex.box["ShipoutBox"]) }
               }
           }
      }
  }
%    \end{macrocode}
% \subsection{Commands}
% \begin{macro}{\@@_mc_if_in:,\tag_mc_if_in:}
% This tests, if we are in an mc, for attributes
% this means to check against a number.
%    \begin{macrocode}
\prg_new_conditional:Nnn \@@_mc_if_in: {p,T,F,TF}
  {
    \int_compare:nNnTF { -2147483647 }={ \lua_now:e{ltx.@@.func.get_mc_type_attr()} }
      { \prg_return_false:  }
      { \prg_return_true: }
  }

\prg_new_eq_conditional:NNn \tag_mc_if_in: \@@_mc_if_in: {p,T,F,TF}
%    \end{macrocode}
% \end{macro}
%
% \begin{macro}
%  {
%     \@@_mc_lua_set_mc_type_attr:n
%    ,\@@_mc_lua_set_mc_type_attr:o
%    ,\@@_mc_lua_unset_mc_type_attr:
%  }
% This takes a tag name, and sets the attributes to the related number.
% It is not decided yet if this will be global or local, see the global-mc option.
%    \begin{macrocode}
\cs_new:Nn \@@_mc_lua_set_mc_type_attr:n % #1 is a tag name
  {
 %   \@@_attribute_set:Nn \l_@@_mc_type_attr
%      {
%        \lua_now:e { ltx.@@.func.output_num_from ("#1") }
%      }
    \lua_now:e {ltx.@@.func.set_mc_type_attr("#1")}
    \@@_attribute_set:Nn \l_@@_mc_cnt_attr  { \@@_get_mc_abs_cnt: }
  }

\cs_generate_variant:Nn\@@_mc_lua_set_mc_type_attr:n { o }

\cs_new:Nn \@@_mc_lua_unset_mc_type_attr:
  {
    %\@@_attribute_unset:N \l_@@_mc_type_attr
    \lua_now:e {ltx.@@.func.unset_mc_type_attr("#1")}
    \@@_attribute_unset:N \l_@@_mc_cnt_attr
  }

%    \end{macrocode}
% \end{macro}
% \begin{macro}
%  {
%    \@@_mc_insert_mcid_kids:n
%   ,\@@_mc_insert_mcid_single_kids:n
%  }
% These commands will in the finish code replace the dummy for a mc by the real mcid kids
% we need a variant for the case that it is the only kid, to get the array right
%    \begin{macrocode}
\cs_new:Nn \@@_mc_insert_mcid_kids:n
  {
    \lua_now:e { ltx.@@.func.mc_insert_kids (#1,0) }
  }

\cs_new:Nn \@@_mc_insert_mcid_single_kids:n
  {
    \lua_now:e {ltx.@@.func.mc_insert_kids (#1,1) }
  }
%    \end{macrocode}
% \end{macro}
%
%\begin{macro}{\@@_mc_handle_stash:n,\@@_mc_handle_stash:o}
% This is the lua variant for the command to
% put an mcid absolute number in the current structure.
%    \begin{macrocode}
\cs_new:Nn \@@_mc_handle_stash:n %1 mcidnum
  {
    \@@_check_mc_used:n { #1 }
    \seq_gput_right:cn % Don't fill a lua table due to the command in the item,
                       % so use the kernel command
      { g_@@_struct_kids_\g_@@_struct_stack_current_tl _seq }
      {
        \@@_mc_insert_mcid_kids:n {#1}%
      }
    \lua_now:e
      {
        ltx.@@.func.store_struct_mcabs
          (
            \g_@@_struct_stack_current_tl,#1
          )
      }
    \prop_gput:Nxx
      \g_@@_mc_parenttree_prop
      { #1 }
      { \g_@@_struct_stack_current_tl }
  }

\cs_generate_variant:Nn \@@_mc_handle_stash:n { o }
%    \end{macrocode}
% \end{macro}
%
% \begin{macro}{\tag_mc_begin:n}
% This is the lua version of the user command.
% Unlike the generic version there is currently no
% group as the attribute is set locally.
% This means one must be careful with the keys.
% If the attribute is global again, wen can change this.
%    \begin{macrocode}
\cs_new_protected:Nn \tag_mc_begin:n
  {
    \@@_check_if_active_mc:T
      {
        %\group_begin:
        %\@@_check_mc_if_nested:
        %\bool_gset_true:N \g_@@_in_mc_bool
        \bool_set_false:N\l_@@_mc_artifact_bool
        \tl_clear:N \l_@@_mc_key_properties_tl
        \int_gincr:N \c@g_@@_MCID_abs_int
        \keys_set:nn { @@ / mc }{ label={}, #1 }
        %check that a tag or artifact has been used
        \@@_check_mc_tag:N \l_@@_mc_key_tag_tl
        %set the attributes:
        \@@_mc_lua_set_mc_type_attr:o  { \l_@@_mc_key_tag_tl }
        \bool_if:NF \l_@@_mc_artifact_bool
          { % store the absolute num name in a label:
            \tl_if_empty:NF {\l_@@_mc_key_label_tl}
              {
                \exp_args:NV
                 \@@_mc_handle_mc_label:n  \l_@@_mc_key_label_tl
              }
           % if not stashed record the absolute number
            \bool_if:NF \l_@@_mc_key_stash_bool
              {
                \exp_args:Nx \@@_mc_handle_stash:n { \@@_get_mc_abs_cnt: }
              }
          }
        %\group_end:
     }
  }
%    \end{macrocode}
% \end{macro}
% \begin{macro}{\tag_mc_end:,\tag_mc_use:n}
% TODO: check how the use command must be guarded.
%    \begin{macrocode}
\cs_new_protected:Nn \tag_mc_end:
  {
    \@@_check_if_active_mc:T
      {
        %\@@_check_mc_if_open:
        %\bool_gset_false:N \g_@@_in_mc_bool
        \bool_set_false:N\l_@@_mc_artifact_bool
        \@@_mc_lua_unset_mc_type_attr:
        \tl_set:Nn  \l_@@_mc_key_tag_tl { }
        \tl_gset:Nn \g_@@_mc_key_tag_tl { }
      }
  }

\cs_new_protected:Nn \tag_mc_use:n %#1: label name
  {
    \tl_set:Nx  \l_@@_tmpa_tl { \@@_ref_value:enn{tagpdf-#1}{tagmcabs}{} }
    \tl_if_empty:NTF\l_@@_tmpa_tl
      {
        \msg_warning:nnn {tag} {mc-label-unknown} { #1 }
      }
      {
        \@@_mc_handle_stash:o { \l_@@_tmpa_tl }
      }
  }
%    \end{macrocode}
% \end{macro}
%
% \begin{macro}{ \@@_get_data_mc_tag: }
% The command to retrieve the current mc tag. When we change to global
% this should use a global variable too!!
%    \begin{macrocode}
\cs_new:Npn \@@_get_data_mc_tag: { \l_@@_mc_key_tag_tl }
%    \end{macrocode}
% \end{macro}

% \subsection{Key definitions}
% \begin{macro}
%   {
%    tag,raw,
%    alttext,alttext-o,
%    actualtext,actualtext-o,
%    label,artifact
%   }
%   TODO: check conversion, check if local/global setting is right.
%    \begin{macrocode}
\keys_define:nn { @@ / mc }
  {
    tag .code:n = %
      {%%????????? \pdfescapename??
        \tl_set:Nx  \l_@@_mc_key_tag_tl { #1 }
        \tl_gset:Nx  \g_@@_mc_key_tag_tl { #1 }
        \lua_now:e
          {
            ltx.@@.func.store_mc_data(\@@_get_mc_abs_cnt:,"tag","#1")
          }
      },
    raw .code:n =
      {
        \tl_put_right:Nx \l_@@_mc_key_properties_tl { #1 }
        \lua_now:e
          {
            ltx.@@.func.store_mc_data(\@@_get_mc_abs_cnt:,"raw","#1")
          }
      },
    alttext .code:n      = % Alt property
      {
        \str_set_convert:Nnon
          \l_@@_tmpa_str
          { #1 }
          { default }
          { utf16/hex }
        \tl_put_right:Nn \l_@@_mc_key_properties_tl { /Alt~< }
        \tl_put_right:No \l_@@_mc_key_properties_tl { \l_@@_tmpa_str>~ }
        \lua_now:e
          {
            ltx.@@.func.store_mc_data
              (
                 \@@_get_mc_abs_cnt:,"alt","/Alt~<\str_use:N \l_@@_tmpa_str>"
              )
          }
      },
    alttext-o .code:n      = % Alt property
      {
        \str_set_convert:Noon
          \l_@@_tmpa_str
          { #1 }
          { default }
          { utf16/hex }
        \tl_put_right:Nn \l_@@_mc_key_properties_tl { /Alt~< }
        \tl_put_right:No \l_@@_mc_key_properties_tl { \l_@@_tmpa_str>~ }
        \lua_now:e
          {
            ltx.@@.func.store_mc_data
              (
                \@@_get_mc_abs_cnt:,"alt","/Alt~<\str_use:N \l_@@_tmpa_str>"
              )
          }
      },
    actualtext .code:n      = % Alt property
      {
        \str_set_convert:Nnon
          \l_@@_tmpa_str
          { #1 }
          { default }
          { utf16/hex }
        \tl_put_right:Nn \l_@@_mc_key_properties_tl { /Alt~< }
        \tl_put_right:No \l_@@_mc_key_properties_tl { \l_@@_tmpa_str>~ }
        \lua_now:e
          {
            ltx.@@.func.store_mc_data
              (
                \@@_get_mc_abs_cnt:,"actualtext","/ActualText~<\str_use:N \l_@@_tmpa_str>"
              )
         }
      },
    actualtext-o .code:n      = % Alt property
      {
        \str_set_convert:Noon
          \l_@@_tmpa_str
          { #1 }
          { default }
          { utf16/hex }
        \tl_put_right:Nn \l_@@_mc_key_properties_tl { /Alt~< }
        \tl_put_right:No \l_@@_mc_key_properties_tl { \l_@@_tmpa_str>~ }
        \lua_now:e
          {
            ltx.@@.func.store_mc_data
              (
                \@@_get_mc_abs_cnt:,
                "actualtext",
                "/ActualText~<\str_use:N \l_@@_tmpa_str>"
              )
          }
      },
    label .code:n =
      {
        \tl_set:Nn\l_@@_mc_key_label_tl { #1 }
        \lua_now:e
          {
            ltx.@@.func.store_mc_data
              (
                \@@_get_mc_abs_cnt:,"label","#1"
              )
          }
      },
    __artifact-store .code:n =
      {
        \lua_now:e
          {
            ltx.@@.func.store_mc_data
              (
                \@@_get_mc_abs_cnt:,"artifact","#1"
              )
          }
      },
    artifact .code:n       =
      {
        \exp_args:Nnx
          \keys_set:nn
            { @@ / mc}
            { __artifact-bool, __artifact-type=#1, tag=Artifact }
        \exp_args:Nnx
          \keys_set:nn
            { @@ / mc }
            { __artifact-store=\l_@@_mc_artifact_type_tl }
      },
    artifact .default:n    = { notype }
  }

%</luamode>
%    \end{macrocode}
% \end{macro}
%
% \end{implementation}
% \PrintIndex

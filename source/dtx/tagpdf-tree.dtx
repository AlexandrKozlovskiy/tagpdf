% \iffalse meta-comment
%
%% File: tagpdf-tree.dtx
%
% Copyright (C) 2019-2021 Ulrike Fischer
%
% It may be distributed and/or modified under the conditions of the
% LaTeX Project Public License (LPPL), either version 1.3c of this
% license or (at your option) any later version.  The latest version
% of this license is in the file
%
%    https://www.latex-project.org/lppl.txt
%
% This file is part of the "tagpdf bundle" (The Work in LPPL)
% and all files in that bundle must be distributed together.
%
% -----------------------------------------------------------------------
%
% The development version of the bundle can be found at
%
%    https://github.com/u-fischer/tagpdf
%
% for those people who are interested.
%
%<*driver>
\RequirePackage{pdfmanagement-testphase}
\DeclareDocumentMetadata{}
\makeatletter
\declare@file@substitution{doc.sty}{doc-v3beta.sty}
\makeatother
\documentclass{l3doc}
\usepackage{array,booktabs,caption}
\hypersetup{pdfauthor=Ulrike Fischer,
 pdftitle=tagpdf-tree module (tagpdf)}
\begin{document}
  \DocInput{\jobname.dtx}
\end{document}
%</driver>
% \fi
% \title{^^A
%   The \pkg{tagpdf-tree} module\\ Commands trees and main dictionaries   ^^A
%   \\ Part of the tagpdf package
% }
%
% \author{^^A
%  Ulrike Fischer\thanks
%    {^^A
%      E-mail:
%        \href{mailto:fischer@troubleshooting-tex.de}
%          {fischer@troubleshooting-tex.de}^^A
%    }^^A
% }
%
% \date{Version 0.9, released 2021-06-29}
% \maketitle
% \begin{implementation}
%    \begin{macrocode}
%<@@=tag>
%<*header>
\ProvidesExplPackage {tagpdf-tree-code} {2021-06-29} {0.9}
 {part of tagpdf - code related to writing trees and dictionaries to the pdf}
%</header>
%    \end{macrocode}
% \section{Trees, pdfmanagement and finalization code}
%
% The code to finish the structure is in a hook.
% This will perhaps at the end be a kernel hook.
% TODO check right place for the code
% The pdfmanagement code is the kernel hook after
% shipout/lastpage so all code affecting it should be before.
% Objects can be written later, at least in pdf mode.
%    \begin{macrocode}
%<*package>
\hook_gput_code:nnn{begindocument}{tagpdf}
  {
    \bool_if:NT \g_@@_active_tree_bool
      {
        \sys_if_output_pdf:TF
          {
            \AddToHook{enddocument/end} { \@@_finish_structure: }
          }
          {
            \AddToHook{shipout/lastpage} { \@@_finish_structure: }
          }
      }
  }
%    \end{macrocode}
% \subsection{Catalog: MarkInfo and StructTreeRoot}
% The StructTreeRoot and the MarkInfo entry must be added to the catalog.
% We do it late so that we can win, but before the pdfmanagement hook.
% \begin{macro}{@@/struct/0}
% This is the object for the root object, the StructTreeRoot
%    \begin{macrocode}
\pdf_object_new:nn { @@/struct/0 }{ dict }
%    \end{macrocode}
% \end{macro}
%    \begin{macrocode}
\hook_gput_code:nnn{shipout/lastpage}{tagpdf}
  {
    \bool_if:NT \g_@@_active_tree_bool
      {
        \pdfmanagement_add:nnn { Catalog / MarkInfo } { Marked } { true }
        \pdfmanagement_add:nnx
          { Catalog }
          { StructTreeRoot }
          { \pdf_object_ref:n { @@/struct/0 } }
      }
  }
%    \end{macrocode}
%
% \subsection{Writing structure elements}
% The following commands are needed to write out the structure.
% \begin{macro}{\@@_tree_write_structtreeroot:}
% This writes out the root object.
%    \begin{macrocode}
\cs_new_protected:Npn \@@_tree_write_structtreeroot:
  {
     \@@_prop_gput:cnx
       { g_@@_struct_0_prop }
       { ParentTree }
       { \pdf_object_ref:n { @@/tree/parenttree } }
     \@@_prop_gput:cnx
       { g_@@_struct_0_prop }
       { RoleMap }
       { \pdf_object_ref:n { @@/tree/rolemap } }
     \@@_struct_write_obj:n { 0 }
  }
%    \end{macrocode}
% \end{macro}
%
% \begin{macro}{\@@_tree_write_structelements:}
% This writes out the other struct elems, the absolute number is in the counter
%    \begin{macrocode}
\cs_new_protected:Npn \@@_tree_write_structelements:
  {
    \int_step_inline:nnnn {1}{1}{\c@g_@@_struct_abs_int}
      {
        \@@_struct_write_obj:n { ##1 }
      }
  }
%    \end{macrocode}
% \end{macro}
%
% \subsection{ParentTree}
% \begin{macro}{@@/tree/parenttree}
% The object which will hold the parenttree
%    \begin{macrocode}
\pdf_object_new:nn { @@/tree/parenttree }{ dict }
%    \end{macrocode}
% \end{macro}
% The ParentTree maps numbers to objects or (if the number represents a page)
% to arrays of objects. The numbers refer to two dictinct types of entries:
% page streams and real objects like annotations.
% The numbers must be distinct and ordered.
% So we rely on abspage for the pages and put the real objects at the end.
% We use a counter to have a chance to get the correct number
% if code is processed twice.
%
% \begin{macro}{\c@g_@@_parenttree_obj_int}
%  This is a counter for the real objects. It starts at the absolute last page
%  value. It relies on l3ref.
%    \begin{macrocode}
\newcounter  { g_@@_parenttree_obj_int }
\hook_gput_code:nnn{begindocument}{tagpdf}
  {
    \int_gset:Nn
      \c@g_@@_parenttree_obj_int
      { \@@_ref_value_lastpage:nn{abspage}{100}  }
  }
%    \end{macrocode}
% \end{macro}
% We store the number/object references in a tl-var. If more structure is needed
% one could switch to a seq.
%  \begin{variable}{ \g_@@_parenttree_objr_tl }
%    \begin{macrocode}
\tl_new:N \g_@@_parenttree_objr_tl
%    \end{macrocode}
% \end{variable}
%
%  \begin{macro}{ \@@_parenttree_add_objr:nn }
%  This command stores a StructParent number and a objref into the tl var.
%  This is only for objects like annotations, pages are handled elsewhere.
%    \begin{macrocode}
\cs_new_protected:Npn \@@_parenttree_add_objr:nn #1 #2 %#1 StructParent number, #2 objref
  {
    \tl_gput_right:Nx \g_@@_parenttree_objr_tl
      {
        #1 \c_space_tl #2 ^^J
      }
  }
%    \end{macrocode}
% \end{macro}
%
% \begin{variable}{\l_@@_parenttree_content_tl}
% A tl-var which will get the page related parenttree content.
%    \begin{macrocode}
\tl_new:N \l_@@_parenttree_content_tl
%    \end{macrocode}
% \end{variable}
%
% \begin{macro}{\@@_tree_fill_parenttree:}
% This is the main command to assemble the page related entries of the parent tree.
% It wanders through the pages and the mcid numbers and collects all mcid of one page.
%    \begin{macrocode}

\cs_new_protected:Npn \@@_tree_fill_parenttree:
  {
    \int_step_inline:nnnn{1}{1}{\@@_ref_value_lastpage:nn{abspage}{-1}} %not quite clear if labels are needed. See lua code
      { %page ##1
        \prop_clear:N \l_@@_tmpa_prop
        \int_step_inline:nnnn{1}{1}{\@@_ref_value_lastpage:nn{tagmcabs}{-1}}
          {
            %mcid####1
            \int_compare:nT
              {\@@_ref_value:enn{mcid-####1}{tagabspage}{-1}=##1} %mcid is on current page
              {% yes
                \prop_put:Nxx
                  \l_@@_tmpa_prop
                  {\@@_ref_value:enn{mcid-####1}{tagmcid}{-1}}
                  {\prop_item:Nn \g_@@_mc_parenttree_prop {####1}}
              }
          }
        \tl_put_right:Nx\l_@@_parenttree_content_tl
          {
            \int_eval:n {##1-1}\c_space_tl
            [\c_space_tl %]
          }
        \int_step_inline:nnnn
          {0}
          {1}
          { \prop_count:N \l_@@_tmpa_prop -1 }
          {
            \prop_get:NnNTF \l_@@_tmpa_prop {####1} \l_@@_tmpa_tl
              {% page#1:mcid##1:\l_@@_tmpa_tl :content
                \tl_put_right:Nx \l_@@_parenttree_content_tl
                  {
                    \pdf_object_if_exist:eT { @@/struct/\l_@@_tmpa_tl }
                     {
                       \pdf_object_ref:e { @@/struct/\l_@@_tmpa_tl }
                     }
                    \c_space_tl
                  }
              }
              {
                \msg_warning:nn { tag } {tree-mcid-index-wrong}
              }
          }
        \tl_put_right:Nn
          \l_@@_parenttree_content_tl
          {%[
            ]^^J
          }
      }
  }
%    \end{macrocode}
% \end{macro}
%
% \begin{macro}{\@@_tree_lua_fill_parenttree:}
% This is a special variant for luatex.
% lua mode must/can do it differently.
%    \begin{macrocode}
\cs_new_protected:Npn \@@_tree_lua_fill_parenttree:
  {
    \tl_set:Nn \l_@@_parenttree_content_tl
      {
        \lua_now:e
          {
            ltx.@@.func.output_parenttree
              (
                \int_use:N\g_shipout_readonly_int
              )
          }
      }
  }
%    \end{macrocode}
% \end{macro}
%
% \begin{macro}{\@@_tree_write_parenttree:}
% This combines the two parts and writes out the object.
% TODO should the check for lua be moved into the backend code?
%    \begin{macrocode}
\cs_new_protected:Npn \@@_tree_write_parenttree:
  {
    \bool_if:NTF \g_@@_mode_lua_bool
      {
        \@@_tree_lua_fill_parenttree:
      }
      {
        \@@_tree_fill_parenttree:
      }
    \tl_put_right:NV \l_@@_parenttree_content_tl\g_@@_parenttree_objr_tl
    \pdf_object_write:nx  { @@/tree/parenttree }
      {
        /Nums\c_space_tl [\l_@@_parenttree_content_tl]
      }
  }
%    \end{macrocode}
% \end{macro}
%
% \subsection{Rolemap dictionary}
% The Rolemap dictionary describes relations between new tags and standard types.
% The main part here is handled in the role module, here we only define the
% command which writes it to the PDF.
% \begin{variable}{@@/tree/rolemap}
% At first we reserve again an object.
%    \begin{macrocode}
\pdf_object_new:nn { @@/tree/rolemap }{ dict }
%    \end{macrocode}
% \end{variable}
%
% \begin{macro}{\@@_tree_write_rolemap:}
% This writes out the rolemap, basically it simply pushes out
% the dictionary which has been filled in the role module.
%    \begin{macrocode}
\cs_new_protected:Npn \@@_tree_write_rolemap:
  {
    \pdf_object_write:nx  { @@/tree/rolemap }
      {
        \pdfdict_use:n{g_@@_role/RoleMap_dict}
      }
  }
%    \end{macrocode}
% \end{macro}
%
% \subsection{Classmap dictionary}
% Classmap and attributes are setup in the struct module, here is only the
% code to write it out. It should only done if values have been used.
% \begin{macro}{\@@_tree_write_classmap:}
%    \begin{macrocode}
\cs_new_protected:Npn \@@_tree_write_classmap:
  {
    \tl_clear:N \l_@@_tmpa_tl
    \seq_gremove_duplicates:N \g_@@_attr_class_used_seq
    \seq_set_map:NNn \l_@@_tmpa_seq \g_@@_attr_class_used_seq
      {
        /##1\c_space_tl
        <<
          \prop_item:Nn
            \g_@@_attr_entries_prop
            {##1}
        >>
      }
    \tl_set:Nx \l_@@_tmpa_tl
      {
        \seq_use:Nn
          \l_@@_tmpa_seq
          { \iow_newline: }
      }
    \tl_if_empty:NF
      \l_@@_tmpa_tl
      {
        \pdf_object_new:nn { @@/tree/classmap }{ dict }
        \pdf_object_write:nx
          { @@/tree/classmap }
          { \l_@@_tmpa_tl }
        \@@_prop_gput:cnx
          { g_@@_struct_0_prop }
          { ClassMap }
          { \pdf_object_ref:n { @@/tree/classmap }  }
      }
  }
%    \end{macrocode}
% \end{macro}
% \subsection{Namespaces}
% Namespaces are handle in the role module, here is the code to write them out.
% Namespaces are only relevant for pdf2.0 but we don't care, it doesn't harm.
% \begin{variable}{@@/tree/namespaces}
%    \begin{macrocode}
\pdf_object_new:nn{ @@/tree/namespaces }{array}
%    \end{macrocode}
% \end{variable}
% \begin{macro}{\@@_tree_write_namespaces:}
%    \begin{macrocode}
\cs_new_protected:Npn \@@_tree_write_namespaces:
  {
    \prop_map_inline:Nn \g_@@_role_NS_prop
      {
        \pdfdict_if_empty:nF {g_@@_role/RoleMapNS_##1_dict}
          {
            \pdf_object_write:nx {@@/RoleMapNS/##1}
              {
                \pdfdict_use:n {g_@@_role/RoleMapNS_##1_dict}
              }
            \pdfdict_gput:nnx{g_@@_role/Namespace_##1_dict}
              {RoleMapNS}{\pdf_object_ref:n {@@/RoleMapNS/##1}}
          }
        \pdf_object_write:nx{tag/NS/##1}
          {
             \pdfdict_use:n {g_@@_role/Namespace_##1_dict}
          }
      }
    \pdf_object_write:nx {@@/tree/namespaces}
      {
        \prop_map_tokens:Nn \g_@@_role_NS_prop{\use_ii:nn}
      }
  }
%    \end{macrocode}
% \end{macro}
% \subsection{Finishing the structure}
% This assembles the various parts.
% TODO (when tabular are done or if someone requests it): IDTree
%  \begin{macro}{ \@@_finish_structure: }
%    \begin{macrocode}
\cs_new_protected:Npn \@@_finish_structure:
  {
    \bool_if:NT\g_@@_active_tree_bool
      {
        \hook_use:n {tagpdf/finish/before}
        \@@_tree_write_parenttree:
        \@@_tree_write_rolemap:
        \@@_tree_write_classmap:
        \@@_tree_write_namespaces:
        \@@_tree_write_structelements: %this is rather slow!!
        \@@_tree_write_structtreeroot:
      }
  }
%    \end{macrocode}
% \end{macro}
%
% \subsection{StructParents entry for Page}
% We need to add to the Page resources the |StructParents| entry, this is simply the
% absolute page number.
%    \begin{macrocode}
\hook_gput_code:nnn{begindocument}{tagpdf}
  {
    \bool_if:NT\g_@@_active_tree_bool
      {
       \hook_gput_code:nnn{shipout/before} { tagpdf/structparents }
         {
           \pdfmanagement_add:nnx
             { Page }
             { StructParents }
             { \int_eval:n { \g_shipout_readonly_int} }
         }
      }
  }
%</package>
%    \end{macrocode}
% \end{implementation}
% \PrintIndex

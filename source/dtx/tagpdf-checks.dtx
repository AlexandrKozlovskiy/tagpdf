% \iffalse meta-comment
%
%% File: tagpdf-checks.dtx
%
% Copyright (C) 2019-2021 Ulrike Fischer
%
% It may be distributed and/or modified under the conditions of the
% LaTeX Project Public License (LPPL), either version 1.3c of this
% license or (at your option) any later version.  The latest version
% of this license is in the file
%
%    https://www.latex-project.org/lppl.txt
%
% This file is part of the "tagpdf bundle" (The Work in LPPL)
% and all files in that bundle must be distributed together.
%
% -----------------------------------------------------------------------
%
% The development version of the bundle can be found at
%
%    https://github.com/u-fischer/tagpdf
%
% for those people who are interested.
%
%
%<*driver>
\RequirePackage{pdfmanagement-testphase}
\DeclareDocumentMetadata{}
\makeatletter
\declare@file@substitution{doc.sty}{doc-v3beta.sty}
\makeatother
\documentclass{l3doc}
\usepackage{array,booktabs,caption}
\hypersetup{pdfauthor=Ulrike Fischer,
 pdftitle=tagpdf-checks module (tagpdf)}
\begin{document}
  \DocInput{\jobname.dtx}
\end{document}
%</driver>
% \fi
% \title{^^A
%   The \pkg{tagpdf-checks} module\\ Messages and check code   ^^A
%   \\ part of the tagpdf package
% }
%
% \author{^^A
%  Ulrike Fischer\thanks
%    {^^A
%      E-mail:
%        \href{mailto:fischer@troubleshooting-tex.de}
%          {fischer@troubleshooting-tex.de}^^A
%    }^^A
% }
%
% \date{Version 0.82, released 2021-06-14}
% \maketitle
% \begin{documentation}
% \begin{function}[pTF,EXP]{\tag_if_active:}
% This command tests if tagging is active. It only gives true if all tagging has
% been activated, \emph{and} if tagging hasn't been stopped locally.
% \end{function}
% \end{documentation}
% \begin{implementation}
%    \begin{macrocode}
%<@@=tag>
%<*checks>
\ProvidesExplPackage {tagpdf-checks-code} {2021-06-14} {0.82}
 {part of tagpdf - code related to checks, conditionals, debugging and messages}
%    \end{macrocode}
% \section{Messages}
% \subsection{Messages related to mc-chunks}
% \begin{macro}{mc-nested}
% This message is issue is a mc is opened before the previous has been closed.
% This is not relevant for luamode, as the attributes don't care about this.
% It is used in the |\@@_check_mc_if_nested:| test.
%    \begin{macrocode}
\msg_new:nnn { tag } {mc-nested} { nested~marked~content~found~-~mcid~#1 }
%    \end{macrocode}
% \end{macro}
% \begin{macro}{mc-tag-missing}
% If the tag is missing
%    \begin{macrocode}
\msg_new:nnn { tag } {mc-tag-missing} { required~tag~missing~-~mcid~#1 }
%    \end{macrocode}
% \end{macro}
% \begin{macro}{mc-label-unknown}
% If the label of a mc that is used in another place is not known (yet)
%    \begin{macrocode}
\msg_new:nnn { tag } {mc-label-unknown} { label~#1~unknown~-~rerun }
%    \end{macrocode}
% \end{macro}
% \begin{macro}{mc-used-twice}
% An mc-chunk can be inserted only in one structure. This indicates wrong coding
% and so should at least give a warning.
%    \begin{macrocode}
\msg_new:nnn { tag } {mc-used-twice} { mc~#1~has~been~already~used }
%    \end{macrocode}
% \end{macro}
% \begin{macro}{mc-not-open}
% This is issued if a |\tag_mc_end:| is issued wrongly, wrong coding.
%    \begin{macrocode}
\msg_new:nnn { tag } {mc-not-open} { there~is~no~mc~to~end~at~#1 }
%    \end{macrocode}
% \end{macro}
% \begin{macro}{mc-pushed,mc-popped}
% Informational messages about mc-pushing.
%    \begin{macrocode}
\msg_new:nnn { tag } {mc-pushed} { #1~has~been~pushed~to~the~mc~stack}
\msg_new:nnn { tag } {mc-popped} { #1~has~been~removed~from~the~mc~stack }
%    \end{macrocode}
% \end{macro}
%
% \subsection{Messages related to mc-chunks}
% \begin{macro}{struct-no-objnum}
% Should not happen \ldots
%    \begin{macrocode}
\msg_new:nnn { tag } {struct-no-objnum} { objnum~missing~for~structure~#1 }
%    \end{macrocode}
% \end{macro}
% \begin{macro}{struct-faulty-nesting}
% This indicates that there is somewhere one |\tag_struct_end:| too much.
% This should be normally an error.
%    \begin{macrocode}
\msg_new:nnn { tag }
  {struct-faulty-nesting}
  { there~is~no~open~structure~on~the~stack }
%    \end{macrocode}
% \end{macro}
% \begin{macro}{struct-missing-tag}
% A structure must have a tag.
%    \begin{macrocode}
\msg_new:nnn { tag } {struct-missing-tag} { a~structure~must~have~a~tag! }
%    \end{macrocode}
% \end{macro}
% \begin{macro}{struct-used-twice}
%    \begin{macrocode}
\msg_new:nnn { tag } {struct-used-twice}
  { structure~with~label~#1~has~already~been~used}
%    \end{macrocode}
% \end{macro}
% \begin{macro}{struct-label-unknown}
% label is unknown, typically needs a rerun.
%    \begin{macrocode}
\msg_new:nnn { tag } {struct-label-unknown}
  { structure~with~label~#1~is~unknown~rerun}
%    \end{macrocode}
% \end{macro}
% \begin{macro}{struct-show-closing}
% Informational message shown if log-mode is high enough
%    \begin{macrocode}
\msg_new:nnn { tag } {struct-show-closing}
  { closing~structure~#1~tagged~\prop_item:cn{g_@@_struct_#1_prop}{S} }
%    \end{macrocode}
% \end{macro}
%
% \subsection{Attributes}
% Not much yet, as attributes aren't used so much.
% \begin{macro}{attr-unknown}
%    \begin{macrocode}
\msg_new:nnn { tag } {attr-unknown}  { attribute~#1~is~unknown}
%    \end{macrocode}
% \end{macro}
%
% \subsection{Roles}
% \begin{macro}{role-missing,role-unknown,role-unknown-tag}
% Warning message if either the tag or the role is missing
%    \begin{macrocode}
\msg_new:nnn { tag } {role-missing}     { tag~#1~has~no~role~assigned  }
\msg_new:nnn { tag } {role-unknown}     { role~#1~is~not~known  }
\msg_new:nnn { tag } {role-unknown-tag} { tag~#1~is~not~known  }
%    \end{macrocode}
% \end{macro}
% \begin{macro}{role-tag,new-tag}
% Info messages.
%    \begin{macrocode}
\msg_new:nnn { tag } {role-tag}         { mapping~tag~#1~to~role~#2  }
\msg_new:nnn { tag } {new-tag}          { adding~new~tag~#1 }
%    \end{macrocode}
% \end{macro}
%
% \subsection{Miscellaneous}
% \begin{macro}{tree-mcid-index-wrong}
% Used in the tree code, typically indicates the document must be rerun.
%    \begin{macrocode}
\msg_new:nnn { tag } {tree-mcid-index-wrong}
  {something~is~wrong~with~the~mcid--rerun}
%    \end{macrocode}
% \end{macro}
% \begin{macro}{obj-write-num}
% An info message, useful for reporting.
%    \begin{macrocode}
\msg_new:nnn { tag } {obj-write-num} {write~obj~#1~to~pdf}
%    \end{macrocode}
% \end{macro}
% \begin{macro}{sys-no-interwordspace}
% Currently only pdflatex and lualatex have some support for real spaces.
%    \begin{macrocode}
\msg_new:nnn { tag } {sys-no-interwordspace}
  {engine/output~mode~#1~doesn't~support~the~interword~spaces}
%    \end{macrocode}
% \end{macro}
%
% \section{User conditionals}
% \begin{macro}[pTF]{\tag_if_active:}
% This is a test it tagging is active. This allows packages
% to add conditional code.
% The test is true if all booleans, the global and the two local one are true.
%
%    \begin{macrocode}
\prg_new_conditional:Npnn \tag_if_active: { p , T , TF, F }
  {
     \bool_lazy_all:nTF
       {
         {\g_@@_active_struct_bool}
         {\g_@@_active_mc_bool}
         {\g_@@_active_tree_bool}
         {\l_@@_active_struct_bool}
         {\l_@@_active_mc_bool}
       }
       {
         \prg_return_true:
       }
       {
         \prg_return_false:
       }
  }
%    \end{macrocode}
% \end{macro}
%
% \section{Internal checks}
% These are checks used in various places in the code.
%
% \subsection{checks for active tagging}
% \begin{macro}[TF]{\@@_check_if_active_mc:,\@@_check_if_active_struct:}
% Structures must have a tag, so we check if the S entry is in the property.
% It is an error if this is missing. The argument is a number.
%    \begin{macrocode}
\prg_new_conditional:Npnn \@@_check_if_active_mc: {T,F,TF}
  {
    \bool_lazy_and:nnTF { \g_@@_active_mc_bool } { \l_@@_active_mc_bool }
      {
         \prg_return_true:
      }
      {
         \prg_return_false:
      }
  }
\prg_new_conditional:Npnn \@@_check_if_active_struct: {T,F,TF}
  {
    \bool_lazy_and:nnTF { \g_@@_active_struct_bool } { \l_@@_active_struct_bool }
      {
         \prg_return_true:
      }
      {
         \prg_return_false:
      }
  }
%    \end{macrocode}
% \end{macro}
% \subsection{Checks related to stuctures}
% \begin{macro}{\@@_check_structure_has_tag:n}
% Structures must have a tag, so we check if the S entry is in the property.
% It is an error if this is missing. The argument is a number.
%    \begin{macrocode}
\cs_new_protected:Npn \@@_check_structure_has_tag:n #1 %#1 struct num
  {
    \prop_if_in:cnF { g_@@_struct_#1_prop }
      {S}
      {
        \msg_error:nn { tag } {struct-missing-tag}
      }
  }
%    \end{macrocode}
% \end{macro}
% \begin{macro}{\@@_check_structure_tag:N}
% This checks if the name of the tag is known.
%    \begin{macrocode}
\cs_new_protected:Npn \@@_check_structure_tag:N #1
  {
    \prop_if_in:NoF \g_@@_role_tags_prop {#1}
      {
        \msg_warning:nnx { tag } {role-unknown-tag} {#1}
      }
  }
%    \end{macrocode}
% \end{macro}
% \begin{macro}{\@@_check_info_closing_struct:n}
% This info message is issued at a closing structure, the use should be guarded by
% log-level.
%    \begin{macrocode}
\cs_new_protected:Npn \@@_check_info_closing_struct:n #1 %#1 struct num
  {
    \msg_info:nnn { tag } {struct-show-closing} {#1}
  }

\cs_generate_variant:Nn \@@_check_info_closing_struct:n {o,x}
%    \end{macrocode}
% \end{macro}
% \begin{macro}{\@@_check_no_open_struct:}
% This checks if there is an open structure. It should be used when trying to
% close a structure. It errors if false.%
%    \begin{macrocode}
\cs_new_protected:Npn \@@_check_no_open_struct:
  {
    \msg_error:nn { tag } {struct-faulty-nesting}
  }
%    \end{macrocode}
% \end{macro}
% \begin{macro}{\@@_check_struct_used:n}
% This checks if a stashed structure has already been used.
%    \begin{macrocode}
\cs_new_protected:Npn \@@_check_struct_used:n #1 %#1 label
  {
    \prop_get:cnNT
      {g_@@_struct_\@@_ref_value:enn{tagpdfstruct-#1}{tagstruct}{unknown}_prop}
      {P}
      \l_tmpa_tl
      {
        \msg_warning:nnn { tag } {struct-used-twice} {#1}
      }
  }
%    \end{macrocode}
% \end{macro}
%
% \subsection{Checks related to roles}
% \begin{macro}{\@@_check_add_tag_role:nn}
% This check is used when defining a new role mapping.
%    \begin{macrocode}
\cs_new_protected:Npn \@@_check_add_tag_role:nn #1 #2 %#1 tag, #2 role
  {
    \tl_if_empty:nTF {#2}
      {
        \msg_warning:nnn { tag } {role-missing} {#1}
      }
      {
        \prop_get:NnNTF \g_@@_role_tags_prop {#2} \l_tmpa_tl
          {
            \msg_info:nnnn { tag } {role-tag} {#1} {#2}
          }
          {
            \msg_warning:nnn { tag } {role-unknown} {#2}
          }
      }
  }
%    \end{macrocode}
% \end{macro}
%
% \subsection{Check related to mc-chunks}
%
% \begin{macro}{\@@_check_mc_if_nested:,\@@_check_mc_if_open:}
% Two tests if a mc is currently open.
%    \begin{macrocode}
\cs_new_protected:Npn \@@_check_mc_if_nested:
  {
    \@@_mc_if_in:T
      {
        \msg_warning:nnx { tag } {mc-nested} { \@@_get_mc_abs_cnt: }
      }
  }

\cs_new_protected:Npn \@@_check_mc_if_open:
  {
    \@@_mc_if_in:F
      {
        \msg_warning:nnx { tag } {mc-not-open} { \@@_get_mc_abs_cnt: }
      }
  }
%    \end{macrocode}
% \end{macro}
% \begin{macro}{\@@_check_mc_pushed_popped:nn}
% This creates an information message if mc's are pushed or popped.
% The first argument is a word (pushed or popped), the second the tag name.
% With larger log-level the stack is shown too.
%    \begin{macrocode}
\cs_new_protected:Npn \@@_check_mc_pushed_popped:nn #1 #2
  {
    \int_compare:nNnT
      { \l_@@_loglevel_int } ={ 2 }
      { \msg_info:nnx {tag}{mc-#1}{#2} }
    \int_compare:nNnT
      { \l_@@_loglevel_int } > { 2 }
      {
       \msg_warning:nnx {tag}{mc-#1}{#2}
       \seq_log:N \g_@@_mc_stack_seq
      }
  }
%    \end{macrocode}
% \end{macro}
% \begin{macro}{\@@_check_mc_tag:N}
% This checks if the mc has a (known) tag.
%    \begin{macrocode}
\cs_new_protected:Npn \@@_check_mc_tag:N #1  %#1 is var with a tag name in it
  {
    \tl_if_empty:NT #1
      {
        \msg_error:nnx { tag } {mc-tag-missing} { \@@_get_mc_abs_cnt: }
      }
   \prop_if_in:NoF \g_@@_role_tags_NS_prop {#1}
     {
       \msg_warning:nnx { tag } {role-unknown-tag} {#1}
     }
  }
%    \end{macrocode}
% \end{macro}
%
% \begin{variable}{\g_@@_check_mc_used_seq}
% This variable holds the list of used mc. It will
% hopefully be rather short, so checking the seq is not
% to slow.
%    \begin{macrocode}
\seq_new:N \g_@@_check_mc_used_seq
%    \end{macrocode}
% \end{variable}
%
% \begin{macro}{\@@_check_mc_used:n}
% This checks if a mc is used twice.
%    \begin{macrocode}
\cs_new_protected:Npn \@@_check_mc_used:n #1
  {
    \seq_if_in:NnTF \g_@@_check_mc_used_seq {#1}
      {
        \msg_warning:nnn { tag } {mc-used-twice} {#1}
      }
      {
        \seq_gput_right:Nx \g_@@_check_mc_used_seq {#1}
      }
  }
%    \end{macrocode}
% \end{macro}
% \begin{macro}{\@@_check_show_MCID_by_page:}
% This allows to show the mc on a page. Currently unused.
%    \begin{macrocode}
\cs_new_protected:Npn \@@_check_show_MCID_by_page:
  {
    \tl_set:Nx \l_@@_tmpa_tl
      {
        \@@_ref_value_lastpage:nn
          {abspage}
          {-1}
      }
    \int_step_inline:nnnn {1}{1}
      {
        \l_@@_tmpa_tl
      }
      {
        \seq_clear:N \l_tmpa_seq
        \int_step_inline:nnnn
          {1}
          {1}
          {
            \@@_ref_value_lastpage:nn
              {tagmcabs}
              {-1}
          }
          {
            \int_compare:nT
              {
                \@@_ref_value:enn
                  {mcid-####1}
                  {tagabspage}
                  {-1}
                =
                ##1
             }
             {
               \seq_gput_right:Nx \l_tmpa_seq
                 {
                   Page##1-####1-
                   \@@_ref_value:enn
                     {mcid-####1}
                     {tagmcid}
                     {-1}
                 }
             }
          }
          \seq_show:N \l_tmpa_seq
      }
  }
%    \end{macrocode}
% \end{macro}
%
% \subsection{Miscellaneous}
%
% \begin{macro}{\@@_check_record_pdfobj_num:n}
% This writes the object numbers. Not sure if needed, currently unused.
%    \begin{macrocode}
\cs_new_protected:Npn \@@_check_record_pdfobj_num:n #1
  {
    \int_compare:nT { \l_@@_loglevel_int >= 3 }
      {
        \msg_info:nnx { tag } {obj-write-num} {#1}
      }
  }
%</checks>
%    \end{macrocode}
% \end{macro}
% \end{implementation}
% \PrintIndex

% \iffalse meta-comment
%
%% File: tagpdf-roles.dtx
%
% Copyright (C) 2019-2021 Ulrike Fischer
%
% It may be distributed and/or modified under the conditions of the
% LaTeX Project Public License (LPPL), either version 1.3c of this
% license or (at your option) any later version.  The latest version
% of this license is in the file
%
%    https://www.latex-project.org/lppl.txt
%
% This file is part of the "tagpdf bundle" (The Work in LPPL)
% and all files in that bundle must be distributed together.
%
% -----------------------------------------------------------------------
%
% The development version of the bundle can be found at
%
%    https://github.com/u-fischer/tagpdf
%
% for those people who are interested.
%
% \fi
%
%    \begin{macrocode}
%<@@=tag>
%<*roles>
\ProvidesExplPackage {tagpdf-roles-code} {2021/02/23} {0.80}
 {part of tagpdf - code related to roles and structure names}
%    \end{macrocode}
% \section{Variables}
% Tags have both a name (a string) and a number (for the lua attribute).
% Testing a name is easier with a prop, while accessing with a number is
% better done with a seq. So both are used and must be kept in sync if a new
% tag is added.
% \begin{macro}{\g_@@_role_tags_seq,\g_@@_role_tags_prop}
%    \begin{macrocode}
\@@_seq_new:N     \g_@@_role_tags_seq  %to get names from numbers
\@@_prop_new:N    \g_@@_role_tags_prop %to get numbers from names
%    \end{macrocode}
% \end{macro}
% \begin{macro}{\g_@@_role_tags_NS_prop}
% in pdf 2.0 tags belong to a name space. We record this in a prop.
% The keys are the tags, the value shorthands like pdf2, or mathml.
% There is no need to access this from lua, so we use the standard prop commands.
%    \begin{macrocode}
\prop_new:N    \g_@@_role_tags_NS_prop %to namespace info
%    \end{macrocode}
% \end{macro}
% \begin{macro}{\g_@@_role_NS_prop}
% The standard names spaces are the following. The keys are the name
% tagpdf will use,  the urls are the identifier in the namespace object.
% \begin{description}
% \item[mathml] http://www.w3.org/1998/Math/MathML
% \item[pdf2]   http://iso.org/pdf2/ssn
% \item[pdf]    http://iso.org/pdf/ssn (default)
% \end{description}
% More namesspaces are possible and
% their objects references and the ones of the namespaces must be collected
% so that an array can be written to the StructTreeRoot at the end (see tagpdf-tree).
% We use a prop to store also the object reference as it will be needed rather
% often.
%    \begin{macrocode}
\prop_new:N \g_@@_role_NS_prop % collect namespaces
%    \end{macrocode}
% \end{macro}
% \section{Namesspaces}
% The following commands setups a names space. Namespace dictionaries can
% contain an optional |/Schema| and |/RoleMapNS| entry. We only reserve the
% objects but delay the writing to finish code, where we can test if there if the
% keys and the name spaces are actually needed
% \begin{macro}{\@@_role_NS_new:nn}
% This commands setups objects for the name space and its rolemap. It also
% initialize a prop to collect the rolemaps if needed.
% \begin{function}{\@@_role_NS_new:nnn}
%  \begin{syntax}
%   \cs{@@_role_NS_new:nnn}\Arg{shorthand}\Arg{URI-ID}{Schema}
%  \end{syntax}
% \end{function}
% \begin{macro}{\@@_role_NS_new:nnn}
%    \begin{macrocode}
\cs_new_protected:Npn \@@_role_NS_new:nnn #1 #2 #3
  {
    \pdf_object_new:nn {c_@@_role/Namespace_#1_obj}{dict}
    \pdf_object_new:nn {c_@@_role/RoleMapNS_#1_obj}{dict}
    \pdfdict_new:n   {g_@@_role/Namespace_#1_dict}
    \pdfdict_new:n   {g_@@_role/RoleMapNS_#1_dict}
    \pdfdict_gput:nnn{g_@@_role/Namespace_#1_dict}
      {Type}{/Namespace}
    \pdf_string_from_unicode:nnN{utf8/string}{#2}\l_tmpa_str
    \pdfdict_gput:nnx{g_@@_role/Namespace_#1_dict}
      {NS}{\l_tmpa_str}
    %RoleMapNS is added in tree
    \pdfdict_gput:nnx{g_@@_role/Namespace_#1_dict}
      {Schema}{#3}
    \prop_gput:Nnx \g_@@_role_NS_prop {#1}{\pdf_object_ref:n{c_@@_role/Namespace_#1_obj}~}
  }
%    \end{macrocode}
% \end{macro}
% Now we setup the standard names spaces. Currently only if we detect pdf2.0 but
% this will perhaps have to change if the structure code gets to messy.
%    \begin{macrocode}
\bool_lazy_or:nnT
  { \pdf_version_compare_p:Nn > {1.9} }
  { \str_if_eq_p:ee{\pdf_version_major:}{-1} }
  {
    \@@_role_NS_new:nnn {pdf}   {http://iso.org/pdf/ssn}{}
    \@@_role_NS_new:nnn {pdf2}  {http://iso.org/pdf2/ssn}{}
    \@@_role_NS_new:nnn {mathml}{http://www.w3.org/1998/Math/MathML}{}
  }
%    \end{macrocode}
%
% \section{Data}
% In this section we setup the standard data.
% At first the list of structure types. We split them in three lists,
% the tags with which are both in the pdf and pdf2 namespace,
% the one only in pdf and the one with the tags only in pdf2.
% We also define a rolemap for the pdfII only type to pdf so that
% they can always be used.
% \begin{macro}
%   {
%     \c_@@_role_sttags_pdf_pdfII_clist,
%     \c_@@_role_sttags_only_pdf_clist,
%     \c_@@_role_sttags_only_pdfII_clist,
%     \c_@@_role_sttags_pdfII_to_pdf_prop
%   }
%    \begin{macrocode}
%
\clist_const:Nn \c_@@_role_sttags_pdf_pdfII_clist
  {
    Document,   %A complete document. This is the root element of any structure tree containing
                %multiple parts or multiple articles.
    Part,       %A large-scale division of a document.
    Sect,       %A container for grouping related content elements.
    Div,        %A generic block-level element or group of elements
    Caption,    %A brief portion of text describing a table or figure.
    Index,
    NonStruct,  %probably not needed
    H,
    H1,
    H2,
    H3,
    H4,
    H5,
    H6,
    P,
    L,           %list
    LI,          %list item (around label and list item body)
    Lbl,         %list label
    LBody,       %list item body
    Table,
    TR,          %table row
    TH,          %table header cell
    TD,          %table data cell
    THead,       %table header (n rows)
    TBody,       %table rows
    TFoot,       %table footer
    Span,        %generic inline marker
    Link,        %
    Annot,
    Figure,
    Formula,
    Form,
    % ruby warichu etc ..
    Ruby,
    RB,
    RT,
    Warichu,
    WT,
    WP,
    Artifact % only MC-tag ?...
  }

\clist_const:Nn \c_@@_role_sttags_only_pdf_clist
 {
   Art,%A relatively self-contained body of text constituting a single narrative or exposition
   BlockQuote, %A portion of text consisting of one or more paragraphs attributed to someone other
               %than the author of the  surrounding text.
   TOC,        %A list made up of table of contents item entries (structure tag TOCI; see below)
                %and/or other nested table of contents entries
   TOCI,       %An individual member of a table of contents. This entry's children can be any of
               %the following structure  tags:
               %Lbl,Reference,NonStruct,P,TOC
   Index,
   Private,
   Quote,       %inline quote
   Note,        %footnote, endnote. Lbl can be child
   Reference,   %A citation to content elsewhere in the document.
   BibEntry,    %bibentry
   Code
 }

\clist_const:Nn \c_@@_role_sttags_only_pdfII_clist
 {
   DocumentFragment
   ,Aside
   ,H7
   ,H8
   ,H9
   ,H10
   ,Title
   ,FENote
   ,Sub
   ,Em
   ,Strong
   ,Artifact
 }

\prop_const_from_keyval:Nn \c_@@_role_sttags_pdfII_to_pdf_prop
  {
    DocumentFragment = Art, %only 2.0, a fragment from another document.
    Aside = Note,    %only 2.0
    Title = H1,      %only 2.0.
    Sub   = Span,    %only 2.0
    H7    = H6 ,     %only 2.0 and more Hn
    H8    = H6 ,
    H9    = H6 ,
    H10   = H6,
    FENote= Note,     %only 2.0 (footnote/endnote)
    Em    = Span,     %only 2.0
    Strong= Span,     %only 2.0
  }
%    \end{macrocode}
% \end{macro}
% We fill the structure tags in to the seq. We allow all pdf1.7 and pdf2.0,
% and role map if needed the 2.0 tags.
%    \begin{macrocode}
% get tag name from number: \seq_item:Nn \g_@@_role_tags_seq { n }
%\seq_gset_from_clist:NN \g_@@_role_tags_seq \c_@@_role_tags_clist

\clist_map_inline:Nn \c_@@_role_sttags_pdf_pdfII_clist
  {
    \@@_seq_gput_right:Nn \g_@@_role_tags_seq { #1 }
  }
\clist_map_inline:Nn \c_@@_role_sttags_only_pdf_clist
  {
    \@@_seq_gput_right:Nn \g_@@_role_tags_seq { #1 }
  }
\clist_map_inline:Nn \c_@@_role_sttags_only_pdfII_clist
  {
    \@@_seq_gput_right:Nn \g_@@_role_tags_seq { #1 }
  }
%    \end{macrocode}
% For luatex we need a name/number relation.
%    \begin{macrocode}
% get tag number from name: \prop_item:Nn \g_@@_role_tags_prop { name }
\int_step_inline:nnnn { 1 }{ 1 }{ \seq_count:N \g_@@_role_tags_seq }
  {
    \@@_prop_gput:Nxn \g_@@_role_tags_prop
      {
        \seq_item:Nn \g_@@_role_tags_seq  { #1 }
      }
      { #1 }
  }
%    \end{macrocode}
%
% Starting with pdf 2.0 we have to handle the NS key for name spaces.
% new tags and the rolemap
%    \begin{macrocode}
\pdfdict_new:n {g_@@_RoleMap_dict}
\@@_prop_new:N \g_@@_role_rolemap_prop

\cs_new_protected:Nn \@@_role_add_tag:nn %(new) name, reference to old
  {
    \prop_if_in:NnF \g_@@_role_tags_prop {#1}
      {
        \msg_info:nnn { tag }{new-tag}{#1}
        \@@_seq_gput_right:Nn \g_@@_role_tags_seq { #1 }
        \@@_prop_gput:Nnx \g_@@_role_tags_prop    { #1 }
         {
           \seq_count:N \g_@@_role_tags_seq
         }
      }
    \@@_check_add_tag_role:nn {#1}{#2}
    \pdfdict_gput:nnn {g_@@_RoleMap_dict}{#1}{#2}
    \tl_if_empty:nF { #2 }
      {
        \@@_prop_gput:Nnn \g_@@_role_rolemap_prop
          { #1 } { #2 }
      }
  }

%this requires that the version is already set. So if it wanders in the kernel
% it must be delayed.
\bool_lazy_or:nnT
  { \pdf_version_compare_p:Nn < {2.0} }
  { \str_if_eq_p:ee{\pdf_version_major:}{-1} }
  {
     \prop_map_inline:Nn \c_@@_role_sttags_pdfII_to_pdf_prop
       {
         \@@_role_add_tag:nn {#1}{#2}
       }
  }

\cs_generate_variant:Nn \@@_role_add_tag:nn {xx}

\keys_define:nn { @@ / setup }
  {
    add-new-tag .code:n =
     {
       \seq_set_split:Nnn \l_tmpa_seq { / } {#1/}
       \@@_role_add_tag:xx
         {
           \seq_item:Nn \l_tmpa_seq {1}
         }
         {
           \seq_item:Nn \l_tmpa_seq {2}
         }
     }
  }


%</roles>
%    \end{macrocode}

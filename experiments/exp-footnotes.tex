% !Mode:: "TeX:DE:UTF-8:Main"

\documentclass{scrartcl}
\usepackage[english]{babel}
\usepackage{tagpdf}

\tagpdfifpdftexT
 {
  \usepackage[T1]{fontenc}
 }

\tagpdfifluatexT
 {
  \usepackage{fontspec}
  \usepackage{luacode}
 }


\tagpdfsetup{activate-all,log=vv,uncompress} %
%to better see the mc in the pdf:
\tagpdfsetup{add-new-tag = FOOTNOTE/Note}


\pagestyle{empty}

\makeatletter
\long\def\@footnotetext#1{\insert\footins{%
    \reset@font\footnotesize
    \interlinepenalty\interfootnotelinepenalty
    \splittopskip\footnotesep
    \splitmaxdepth \dp\strutbox \floatingpenalty \@MM
    \hsize\columnwidth \@parboxrestore
    \protected@edef\@currentlabel{%
       \csname p@footnote\endcsname\@thefnmark
    }%
    \color@begingroup
    \tagstructbegin{tag=FOOTNOTE}%
      \@makefntext{%
        \rule\z@\footnotesep\tagstructbegin{tag=P}\tagmcbegin{tag=FOOTNOTE}\ignorespaces #1\@finalstrut\strutbox
        \tagmcend\tagstructend}%
     \tagstructend   
    \color@endgroup}}%
    
  
\deffootnote[1em]{1.5em}{1em}{%
\textsuperscript{\tagstructbegin{tag=Lbl}\tagmcbegin{tag=Lbl}\thefootnotemark\tagmcend}\tagstructend}
\deffootnotemark{\textsuperscript{\tagmcbegin{artifact}\thefootnotemark\tagmcend}}
 \makeatletter
 
\begin{document}
% Some tests with mc + footnotes
\tagstructbegin{tag=Document}
\tagstructbegin{tag=P}
\tagmcbegin{tag=P}Hallo\tagmcend
\footnote{Dies ist ein Test}
\tagmcbegin{tag=P}Welt\tagmcend
\tagstructend
\tagstructend

\end{document}

Remarks: this looks more or less okay, but the mix between kernel and KOMA-patches is not so good. 
The xml structure looks like this


<Document>
 <P>yyy
  <FOOTNOTE>
   <Lbl>1</Lbl>
   <P>note</P>
  </FOOTNOTE>
xxx</P>
</Document>

Not quite clear yes if this the wanted result (test with screen reader not really successful).


